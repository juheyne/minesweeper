\chapter{Introduction}

\section{Motivation}
Reinforcement Learning is an interesting field of machine learning. An agent learns which
actions to take in an environment given certain rewards. However, the naive algorithms
struggle with high--dimensional sensory input. One way to bypass this problem is the
usage of handcrafted features. These systems are unfortunately not easily applicable
to other problems and highly depend on the quality of the feature representation. Deep
learning makes it possible to extract higher levels of abstractions from raw sensory input.
These techniques are for example used by Google Deepmind to play seven, and later 49
Atari games with a deep Q{network [mnih2013playing, mnih2015human].
Minesweeper is a game almost everyone, who used Microsoft Windows, knows. 
This makes it an interesting subject for a project applying reinforcement learning. Playing
it with a deep reinforcement learning approach enables to learn more in the fields of reinforcement as well as deep learning.

\section{Minesweeper}
Minesweeper is a video game which was introduced in the $1960$s. 
%TODO vielleicht Geschichte?!
The game is represented by equally looking squares that are arranged on a grid. 
Some of the squares contain mines which are randomly assigned to the squares. 
At the beginning of every game the grid size as well as the number of mines on the grid is known.
The Player can select one of the squares. 
If the square that contains a mine is chosen the game is lost.
Else the square will show a digit from $0$ to $8$.
This digit represents the number of mines that surround the chosen field.
In other words if no field with a mine is adjacent to the chosen one, the chosen one has the digit $0$. 
If all the adjacent squares contain mines the digit will be $8$.
In the original game the player can not only choose to open a square he can also 'flag' or 'unflag' a filed.
This means if the player is sure that a field contains a mine he can flag it to mark that field contains a mine. 
Squares that contain are flagged can not be opened. 
To open a flagged field the player has to unflag it first.

In our implementation we chose not to have the 'flag' or 'unflag' options. This is due to the deadlocks that were created by these actions.

A game is won if all squares without mines are open.